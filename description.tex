\chapter*{Programa da disciplina}

\section*{Descrição}

Esta displicina é uma introdução à conceitos de matemática discreta e estruturas
discretas tal como são utilizadas em Ciência da Computação. As técnicas
apresentadas no curso permitem aos estudantes aplicar o pensamento lógico e
matemático na resolução de problemas. Os tópicos incluem: lógica proposicional
e de predicados, funções, relações, conjuntos, técnicas de demonstração, grafos
e árvores. De acordo a disponibilidade de tempo, os seguintes tópicos também
serão apresentados outros tópicos mais avançados.

\section*{Objectivos}

Ao completar a disciplina, os estudantes deverão capazes de:
\begin{itemize}
  \item Aplicar métodos formais de lógica proposicional e de predicados
  \item Descrever a importância e limitações da lógica de predicados
  \item Utilizar demonstrações lógicas para resolver problemas
  \item Desenvolver algoritmos recursivos baseados em indução matemática
  \item Explicar a terminologia básica das funções, relações e conjuntos
  \item Descrever como métodos formais de lógica simbólica são utilizados para
  modelar algoritmos reais
  \item Perceber os conceitos básicos sobre a teoria dos grafos e algoritmos relacionados\
\end{itemize} 

No geral, espera-se que os estudantes sejam capazes de aplicar estes métodos em
outros tópicos do curso de Ciência da Computação tais como no desenho e análise
de algoritmos e engenharia de \emph{software}.

\section*{Tópicos}
\begin{enumerate}
  \item Lógica formal
  \item Demonstrações, recorrência e análise de algoritmos
  \item Conjuntos e combinatória
  \item Relações, funções e matrizes
  \item Gráfos e árvores
  \item Álgebra de Boole e lógica computacional
\end{enumerate}

\section*{Avaliação}%
\begin{itemize}
  \item Avaliação contínua (Participação, Presença, Exercícios, Provas
  parcelares e Projecto): 40\%
  \item Exame final: 60\%
\end{itemize}

Os exercícios serão fornecidos nas aulas ou publicados no \emph{website} da cadeira. Os estudantes são fortemente encorajados
a resolve-los pois os mesmos irão ajudar a entender melhor os tópicos tratados nas aulas.
%
\section*{Pré-requisitos}

Conhecimentos básicos de lógica e de simbolização matemática.

\section*{Regras}

\begin{itemize}
  \item {Requere-se que os estudantes leiam os acetatos/fascículos antes das
  aulas}
  \item {A participação nas aulas é essencial para a compreensão da matéria. A assistência às aulas é de sua inteira
  responsabilidade}
  \item {Todas as provas e exames são obrigatórios, excepto por razões devidamente justificadas. Uma ausência não justificada, 
  equivale a nota zero na referida avaliação.}
\end{itemize}

\section*{Agenda (sujeita à alterações)}

\begin{table}[H]
	\centering
	\begin{tabular}{lll}%
	\toprule
	\textbf{Semana} & \textbf{Intervalo} & \textbf{Tópicos} \\ 
	\midrule
	1	&	1 - 5 de Agosto	&	--\\
	2	&	8 - 12 de Agosto	&	--\\%[Lógica formal}\\
    3 	&	15 - 19 de Agosto	&	--\\%{Demonstrações}\\
    4	&	22 - 26 de Agosto	&	--\\%Conjuntos\\
    5	&	29 de Agosto - 2 de Setembro	&	--\\%Funções\\
    6	&	5 - 9 de Setembro	& 	--\\%Relações\\
    7	&	12 - 16 de Setembro	&	--\\%Algoritmos\\
    8	&	19 - 23 de Setembro	&	--\\%Indução\\
    9	&	26 - 30 de Setembro &	--\\%Contagem\\
    10	&	3 - 7 de Outubro	& 	--\\%Combinatória\\
    11	&	10 -14 de Outubro	&	--\\%Recursão\\
    12	&	17 - 21 de Outubro	& 	--\\%Grafos\\
    13	&	24 - 28 de Outubro 	&	--\\%{Algoritmos para grafos}\\
    14	&	31 de Outubro - 4 de Novembro	&	--\\%Árvores\\
    15	&	7 - 11 de Novembro	& 	--\\%{Álgebra de Boole}\\
    16  &	14 - 18 de Novembro	&	--\\
    17	&	21 - 25 de Novembro	&	--\\
 	 \bottomrule
 	 \end{tabular}
 	 \centering
\end{table}
%Acrescentar unidades

\subsection*{Feriados e interrupções}

\begin{itemize}
  \item 2 de Novembro: Dia dos Finados
  \item 11 de Novembro: Dia da Independência Nacional
\end{itemize}

\section*{Bibliografia}

\begin{table}[H]
	\begin{tabular}{ll}%
		Título & Fundamentos matemáticos para a Ciência da Computação, 6a. Edição\\
		Autor & Judith L. Gersting\\
	\end{tabular}
\end{table}

\section*{Docente}

\begin{table}[H]
	\begin{tabular}{ll}%
		Nome 			& Dizando Norton \\ 
	    Sala			& CS119, Campus Universitário\\
	    Atendimento 	& Por agendamento \\
	    Telefone		& 919075381\\
	    E-mail			& \url{dizando.norton@gmail.com}\\
	    Website			& \url{www.dizando.me}\\
	    Aulas 			& Consultem o horário para a vossa turma\\
	    Monitor			& Nzuzi Solange (\url{n.solange@outlook.com})\\
	    %				& Quarta-Feira, 7h30 - 9h50\\
	\end{tabular}
\end{table}

\section*{Página da cadeira}

O \emph{link} para página da disciplina será disponibilizado em breve. 
A mesma irá conter anúncios, os acetatos e a sebenta utilizada no decorrer das
aulas. Os estudantes são recomendados a consultar a página regularmente.

%\subsection*{Moodle}
%\subsection*{dizan.do}
%\subsection*{Dropbox}




