\chapter{Grafos}
\label{cap:grafos}

Grafos são estruturas discretas que consistem em vértices, e arestas que
conectam estes vértices. Existem vários tipos de grafos, dependendo da
existência de uma direcção nas arestas, de acordo a possibilidade de várias
arestas se interligarem ao mesmo par de vértices e de acordo a existência de
\emph{loops} ou repetições.
Problemas em quase todas as disciplinas podem resolvidos utilizando modelos de
grafos. Iremos apresentar alguns exemplos para ilustrar como os grafos são
utilizados como modelos numa variedade de aéras. Por exempo, iremos mostrar como
os grafos são utilizados para representar a competição de diferentes espécias
num nicho ecológico, e como os grafos são usados para representar quem
influencia quem numa organização e etc.

Utilizando modelos de grafos, podemos determinar se é possível caminhar todas as
ruas de uma cidade sem psasar por uma rua duas vezes, e podemos determinar o
número de cores necessário para colorar as regiões de um mapa. Grafos podem ser
utilizados para determinar se um circuito pode ser implementado numa placa de
circuitos plana. Podemos distinguir entre dois compostos químicos com a mesma
fórmula molecular mas estruturas diferentes utilizando grafos. Podemos
determinar se dois computadores estão conectados por um \emph{link} de
comunicação utilizando modelos gráficos de redes. Grafos com pesos atribuidos as
suas arestas podem ser utilizados para resolver problemas como encontrar o
caminho mais curto entre duas cidades numa rede de transporte. Neste capítulo
iremos apresentar os conceitos básicos da teoria dos grafos e apresentar alguns
modelos de grafos. Para resolver uma boa parte dos problemas que podem ser
estudados utilizando grafos, iremos apresentar alguns algoritmos de grafos.
Iremos também estudar a complexidade destes algoritmos.

\section{Grafos e Modelos de Grafos}

Começamos com a definição de grafos.
\begin{defn}
\label{def51}
Um \emph{grafo} $G = (V,E)$ consiste em $V$, um conjunto não-vazio de
\emph{vértices} (ou \emph{nós}) e $E$, um conjunto de \emph{arestas}. Cada
aresta possui ou um ou mais vértices associados a esta, chamada de sua
\emph{extremidade}. Diz-se que uma aresta \emph{conecta} as suas extremidades.
Teste
\end{defn}