\chapter{Lógica Formal}
\label{cap:logicaformal}
\newtheorem{thm}{Theorem}[section] % reset theorem numbering for each chapter
\newtheorem{lm}{Theorem}[section]


\newtheorem{defn}[thm]{Definição}
\newtheorem{exmp}[lm]{Exemplo}

As regras da lógica fornecem o significado de expressões matemáticas. Por
exemplo, estas regras nos ajudam a entender e a raciocinar sobre sentenças como
\emph{``Existe um inteiro que é a soma de dois quadrados''} e \emph{``Para todo o
inteiro positivo \emph{n}, a soma dos positivos não maiores que \emph{n} é
$n(n+1)/2$''}. A lógica é a base do raciocínio matemático, e do raciocínio
automatizado. Possui aplicações práticas no desenho de computadores, na
especificação de sistemas, na Inteligêngia Artifical, na programação de computadores, nas linguagens de
programação e outras áreas da Ciência da Computação bem como também em outras
ciências e áreas de estudo.

Para entender a matemática, precisamos perceber o que constitui um argumento
matemático correcto, isto é, uma prova ou demonstração. Uma vez que demonstramos
que uma sentença matemática é verdadeira, nós a chamamos de teorema. Uma
colecção de teoremas sobre um assunto ou tópico constitui o que sabemos sobre tal
tópico. Então, para perceber um tópico matemático, é necessário activamente
construir argumentos matemáticos sobre tal tópico. As demonstrações são
muito comuns em mátematica e também em Ciência da Computação. Elas são
utilizadas para verificar se os programas de computadores produzem a saída
correcta para todos os valores de entrada possíveis, para verificar se os
algoritmos produzem sempre a saída correcta, para determinar a segurança de um
sistema e para criar inteligência artificial. Além disso, sistemas de raciocínio
automatizados foram criados para permitir que os computadores construam as suas
próprias demonstrações.

Neste capítulo, iremos apresentar o que constitui uma sentença matemática
correcta e introduzir as ferramentas para a construção destes argumentos. Iremos
estudar alguns métodos de demonstração e estratégias para a construção de
demonstrações.

\section{Lógica proposicional}
\label{sec:logicaproposicional}

As regras da lógica fornecem significados às sentenças ou argumentos
matemáticos. Estas regras são utilizadas para distinguir um argumento matemático
válido de um inválido.

Para além da sua importância na compreensão do raciocínio matemático, a lógica
possui inúmeras aplicações em Ciência da Computação. Algumas dessas aplicações
serão discutidas ao longo desta secção.

\subsection*{\underline{Proposições}}

Começamos a nossa discussão com uma introdução sobre o elemento básico da lógica
- as proposições. Uma \textbf{proposição} é uma sentença declarativa (isto é,
uma sentença que declara um facto) que pode ser verdadeira ou falsa, mas não
ambas.
%

\begin{exmp}
\label{exem11}
As seguintes sentenças declarativas são exemplos de proposições.
\end{exmp}
\begin{enumerate}
  	\item Luanda é a capital da República de Angola.
  	\item Angola possui 24 províncias.
  	\item $1 + 1 = 2$.
  	\item $1 + 2 < 1$.
\end{enumerate}

As proposições 1 e 3 são verdadeiras, enquanto que as proposições 2 e 4 são
falsas. Algumas sentenças que não são proposições são dadas no Exemplo 2.

%\begin{description}

\begin{exmp}
\label{exem12}
Considere as seguintes sentenças.\end{exmp}
\begin{enumerate}
  	\item Que horas são?
  	\item Leia isto atentamente.
  	\item $x + 1 = 2$.
  	\item $x + y = z$.
\end{enumerate}
%\end{description}

As sentenças 1 e 2 não são proposições porque não são sentenças declarativas. As
sentenças 3 e 4 não são proposições porque ambas não são verdadeiras nem falsas.
Note que as sentenças 3 e 4 podem ser transformadas em proposições se
atribuir-mos valores às variáveis $x$ e $y$.

Utilizamos letras para denotar as \textbf{variáveis proposicionais} (ou
\textbf{variáveis da expressão}), isto é, variáveis que representam proposições,
tal como as letras são utilizadas para denotar valores numéricos. As letras
convencionais utilizadas para variáveis proposicionais são $p, q, r, s, \ldots$.
O \textbf{valor lógico} de uma proposição é verdadeiro, denotado por V, se a
proposição for verdadeira. Consequentemente, o valor lógico de uma
proposição é falso, denotado por F, se a proposição for falsa.

A área da lógica que trata das proposições é chamada de \textbf{cálculo
proposicional} ou \textbf{lógica proposicional}. \emph{TAREFA: Pesquisar sobre
Aristotle e o Cálculo Proposicional}.

Muitas expressões matemáticas são construídas pela combinação de uma ou mais
proposições. Novas proposições, chamadas de \textbf{proposições compostas}, são
formadas a partir de proposições existentes utilizando os operadores lógicos.

\begin{defn}\label{def11}(Negação) Seja $p$ uma proposição. A
\emph{negação} de $p$, denotada por $\lnot p$ (também denotada por $\bar{p}$),
é a expressão\end{defn}

``Não é verdade que $p$''.\\ \\
A proposição $\lnot p$ lê-se ``não $p$''. O valor lógico da negação de $p$,
$\lnot p$, é o oposto do valor lógico de $p$.

\label{exem13}
\begin{exmp}Encontre a negação da proposição\end{exmp}
\emph{``O computador do Miguel possui o Sistema Operativo Linux.''}\\ \\
e a expresse em Português simples.

\begin{description}
\item[Solução] A negação é: \emph{``Não é verdade que o computador do Miguel
possui o Sistema Operativo Linux.''}\\ \\
Esta negação pode ser expressada de forma mais simples como: \emph{``O
computador do Miguel não possui o Sistema Operativo Linux.''}
\end{description}


A tabela \ref{tabela:11} apresenta a \textbf{tabela de verdade} para a negação
de uma proposição $p$. Esta tabela possui uma linha para cada um dos valores
lógicos possíveis para a proposição $p$. Cada linha apresenta o valor lógico de
$\lnot p$ correspondente ao valor lógico de $p$ para esta linha.

\begin{table}[H]
\centering
\begin{tabular}{|c|c|}%
\toprule
\textbf{$p$} & \textbf{$\lnot p$}\\ 
\midrule
V	&	F\\
F	&	V\\
\bottomrule%
\end{tabular}%
\caption{A tabela de verdade da negação de uma proposição.}
\label{tabela:11}
\end{table}

A negação de uma proposição pode ser também considerada como o resultado da
acção do \textbf{operador de negação} numa proposição. O operador de negação
constrói uma nova proposição a partir de uma única proposição. Iremos agora
apresentar os operadores lógicos que são utilizados para formar novas
proposições a partir de duas ou mais proposições existentes. Estes operadores lógicos também
são chamados de \textbf{conectores}.

\label{def12}
\begin{defn}
(Conjunção) Sejam as proposições $p$ e $q$. A
	\emph{conjunção} de $p$ e $q$, denotada por $p \land q$, é a proposição ``$p$
	e $q$''. A conjunção $p \land q$ é verdadeira quando ambos $p$ e $q$ são
	verdadeiros e falsa em caso contrário.
\end{defn}

A tabela \ref{tabela:12} apresenta a tabela de verdade de $p \land q$. Esta
tabela possui uma linha para cada um das quatro possíveis combinações dos valores
lógicos de $p$ e $q$. As quatro linhas correspondem aos pares dos valores
lógicos VV, VF, FV e FF, onde o primeiro valor lógico no par é o valor lógico de $p$ e o
segundo valor lógico é o valor lógico de $q$.

\begin{table}[H]
\centering
\begin{tabular}{|c|c|c|}%
\toprule
\textbf{$p$} & \textbf{$q$} & \textbf{$p \land q$}\\ 
\midrule
V & V & V\\
V &	F & F\\
F &	V & F\\
F &	F & F\\
\bottomrule%
\end{tabular}%
\caption{A tabela de verdade da conjunção de duas proposições.}
\label{tabela:12}
\end{table}


\begin{exmp}
\label{exem14}
Encontre a conjunção das proposições $p$ e $q$ onde $p$ é a proposição \emph{``O
computador da Rebecca possui mais de 16 GB de espaço livre em disco''} e $q$ é a
proposição \emph{``O processador no computador da Rebecca tem uma velocidade maior que 1 GHz.''}.
\end{exmp}
	
\begin{description}
	\item[Solução] A conjunção destas duas proposições, $p \land q$, é a proposição
	\emph{``O computador da Rebecca possui mais de 16 GB de espaço livre em disco
	e o seu processador tem uma velocidade maior que 1 GHz.''} Para esta conjunção
	ser verdadeira ambas as condições dadas devem ser verdadeiras.  É falsa quando
	uma ou ambas as condições são falsas.
\end{description}

\label{def13}
\begin{defn}
(Disjunção) Sejam as proposições $p$ e $q$. A
	\emph{disjunção} de $p$ e $q$, denotada por $p \lor q$, é a proposição ``$p$
	ou $q$''. A disjunção $p \lor q$ é falsa quando ambos $p$ e $q$ são
	falsos, e verdadeira em caso contrário.
\end{defn}

A tabela \ref{tabela:13} apresenta a tabela de verdade de $p \lor q$.

\begin{table}[H]
\centering
\begin{tabular}{|c|c|c|}%
\toprule
\textbf{$p$} & \textbf{$q$} & \textbf{$p \lor q$}\\ 
\midrule
V & V & V\\
V &	F & V\\
F &	V & V\\
F &	F & F\\
\bottomrule%
\end{tabular}%
\caption{A tabela de verdade da disjunção de duas proposições.}
\label{tabela:13}
\end{table}

A utilização do conector \emph{ou} numa disjunção corresponde a uma das
utilizações da palavra \emph{ou} na Língua Portuguesa, nomeadamente,
como um \textbf{ou inclusivo}. A disjunção é verdadeira quando pelo menos uma
das duas proposições é verdadeira. Por exemplo, o ``ou inclusivo'' é utilizado na
expressão:\\ \\
\emph{``Os estudantes aprovados em Análise ou Álgebra podem frequentar esta
aula.''}\\ \\
Aqui, queremos dizer que os estudantes que aprovaram em Análise e Álgebra podem
assistir a esta aula, bem como os estudantes que aprovaram em pelo menos uma
delas. Por outro lado, estaremos a utilizar um ``ou exclusivo'' quando
dizemos:\\
\\
\emph{``Os estudantes aprovados em Análise ou Álgebra, mas não em ambas, podem
frequentar esta aula.''}\\ \\
Aqui, queremos dizer que os estudantes que aprovaram em Análise e Álgebra não
podem assisir a esta aula. Apenas estes que aprovaram em exactamente uma das duas
disciplinas podem assistir à esta aula.

\label{exem15}
\begin{exmp}
Qual é a disjunção das proposições $p$ e $q$ onde $p$ e $q$ são as mesmas
proposições utilizadas no exemplo \ref{exem14}?
\end{exmp}
\begin{description}
	\item[Solução] A disjunção das proposições $p$ e $q$, $p \lor q$, é a
	proposição\\ \\
	\emph{``O computador da Rebecca possui mais de 16 GB de espaço
	livre em disco ou o processador do seu computador tem uma velocidade maior que
	1 GHz.''} \\ \\
	
	Esta proposição é verdadeira quando o computador de Rebecca possuir mais de 16
	GB de espaço livre em disco, o processador possuir uma velocidade maior que 1
	GHz e quando ambas as condições forem verdadeiras. Será falsa quando ambas as
	condições forem falsas.
\end{description}

\begin{defn}
	\label{def14}
	(Ou-Exclusivo ou Disjunção Exclusiva) Sejam as proposições $p$ e $q$. A
	\emph{disjunção exclusiva} de $p$ e $q$, denotada por $p \oplus q$, é a
	proposição que é verdadeira quando exactamente apenas uma entre ``$p$ ou $q$''
	é verdadeira e falsa em caso contrário.
\end{defn}

A tabela de verdade para a disjunção exclusiva de duas proposições é ilustrada
na tabela \ref{tabela:14}.

\begin{table}[H]
\centering
\begin{tabular}{|c|c|c|}%
\toprule
\textbf{$p$} & \textbf{$q$} & \textbf{$p \oplus q$}\\ 
\midrule
V &	V & F\\
V &	F & V\\
F &	V & V\\
F &	F & F\\
\bottomrule%
\end{tabular}%
\caption{A tabela de verdade da disjunção exclusiva de duas proposições.}
\label{tabela:14}
\end{table}

\subsection*{\underline{Sentenças condicionais}}

Iremos agora abordar sobre outras formas importantes de combinar proposições.

\label{def15}
\begin{defn}
(Sentença condicional) Sejam as proposições $p$ e $q$. A \emph{sentença
condicional} de $p \to q$ é a proposição ``se $p$, então $q$.'' A sentença
condicional $p \to q$ é falsa quando $p$ é verdadeiro e $q$ é falso, e
verdadeira em caso contrário. Na sentença condicional $p \to q$, $p$ é chamado
de ``hipótese'' (ou antecedente ou premissa) e $q$ é chamado de conclusão (ou consequente).
\end{defn}

A sentença $p \to q$ é chamada de sentença condicional porque $p \to q$ afirma
que $q$ é verdadeiro caso $p$ se verifique. Uma sentença condicional é também
chamada de \textbf{implicação}. A tabela de verdade para a sentença condicional
$p \to q$ é apresentada na Tabela \ref{tabela:15}. Note que a expressão $p \to
q$ é verdadeira quando ambos $p$ e $q$ são verdadeiros e quando $p$ é falso
(não importa qual seja o valor de $q$).

\begin{table}[H]
\centering
\begin{tabular}{|c|c|c|}%
\toprule
\textbf{$p$} & \textbf{$q$} & \textbf{$p \to q$}\\ 
\midrule
V &	V & V\\
V &	F & F\\
F &	V & V\\
F &	F & V\\
\bottomrule%
\end{tabular}%
\caption{A tabela de verdade da sentença condicional $p \to q$.}
\label{tabela:15}
\end{table}


Devido ao facto de as sentenças condicionais desempenharem um papel
essêncial no raciocínio matemático, existe uma vasta gama de termos utilizados
para expressar $p \to q$. Algumas destes termos são:

\begin{itemize}
  \item ``se $p$, então $q$''
  \item ``$p$ implica $q$''
  \item ``se $p$, $q$''
  \item ``$q$ somente se $p$''
  \item ``$p$ é suficiente para $q$''
  \item ``$q$ se $p$''
  \item ``$q$ quando $p$''
  \item \ldots
\end{itemize}


Uma forma útil para entender uma tabela de verdade de uma sentença condicional é
pensar numa obrigação ou contrato. Por exemplo, a promessa que muitos políticos
fazem quando concorrem para uma determinada eleição é:

\begin{center}\emph{``Se eu for eleito, irei reduzir os impostos.''}\end{center}

Se o político é eleito, os eleitores irão esperar que o político reduza os
impostos. Além disso, se o político não for eleito, então os eleitores não
terão expectativas que o mesmo político reduza os impostos, embora o
político possa ter alguma influência para que tal aconteça. É apenas quando o político é
eleito e não reduz os impostos que os eleitores podem dizer que o político não
cumpriu com a sua promessa de campanha. Este último cenário corresponde ao caso
em que $p$ é verdadeiro mas $q$ é falso em $p \to q$.

\label{exem16}
\begin{exmp}
Seja $p$ a sentença ``Maria aprende Estruturas Discretas'' e
$q$ a sentença ``Maria conseguirá um bom emprego''. Expresse a sentença $p
\to q$ como uma sentença em Português.
\end{exmp}


\begin{description}
	\item[Solução] Da definição de sentenças condicionais, vemos que quando $p$ é a
	expressão ``Maria aprende Estruturas Discretas'' e $q$ a sentença ``Maria
	conseguirá um bom emprego'', $p \to q$ representa a sentença
	\begin{center}\emph{``Se Maria aprende Estruturas Discretas, então ela
	encontrará um bom emprego.''}\end{center}
	
Existem muitas outras formas de expressar esta sentença condicional em
Português. As formais mais naturais poderiam ser:
	\begin{itemize}
	 	\item \emph{``Maria encontrará um bom emprego quando ela aprender Estruturas
Discretas''}.
		\item \emph{``Para Maria encontrar um bom emprego, é suficiente que ela
	aprenda Estruturas Discretas''}.
		\item \emph{``Maria encontrará um bom emprego, ao menos que ela não aprenda
Estruturas Discretas''}.
	\end{itemize}
\end{description}

A forma como definimos as sentenças condicionais é mais geral do que o
significado associado à estas sentenças na Língua Portuguesa. Por exemplo, a
sentença condiconal
\begin{center}\emph{``Se estiver a chover, então iremos à praia''}\end{center}

é uma sentença utilizada na linguagem normal onde existe uma relação entre a
hipótese e a conclusão. Por outro lado, a sentença

\begin{center}\emph{``Se João tem um telemóvel, então $2+3=5$''}\end{center}

é verdadeira pela definição de sentença condicional, porque a sua
conclusão/consequente é verdadeira/o. (O valor lógico da hipótese ficaria sem
efeito). 

A sentença condicional

\begin{center}\emph{``Se João tem um telemóvel, então $2+3=6$''}\end{center}

é verdade se João não possui um telemóvel, embora $2+3=6$ seja falso. Não
utilizaríamos estas duas sentenças condicionais em linguagem natural (excepto
em sarcasmo), porque não existe relação entre a hipótese e a conclusão em
nenhuma das sentenças. O conceito matemático sobre sentenças condicionais é
independente de uma relação de causa-efeito entre uma hipótese e uma conclusão.
A nossa definição de sentença condicional especifica apenas valores lógicos; não
é baseada na utilização dada na língua Portuguesa. A linguagem proposicional é
uma linguagem artificial.

\subsubsection*{Conversa, Contrapositiva e Inversa}

Podemos formar novas sentenças condicionais partindo da sentença condicional $p
\to q$. Em particular, existem três sentenças condicionais relacionadas que
por ocorrerem frequentemente receberam nomes especiais. A proposição $q \to
p$ é chamada de \textbf{conversa} de $p \to q$. A \textbf{contrapositiva} de $p \to q$ é a
proposição $\lnot q \to \lnot p$. A proposição $\lnot p \to \lnot q$ é chamada
de \textbf{inversa} de $p \to q$. Veremos que destas três sentenças condicionais
formadas a partir de $p \to q$ apenas a contrapositiva possui sempre o mesmo
valor que $p \to q$.

Primeiro demonstraremos que a contrapositiva, $\lnot q \to \lnot p$, de uma
sentença condicional $p \to q$ possui sempre o mesmo valor lógico que $p \to
q$. Para verificar isto, note que a contrapositiva é falsa apenas quando $\lnot
p$ é falso e $\lnot q$ é verdadeiro, isto é, quando $p$ é verdadeiro e $q$ é
falso. Agora demonstraremos que nem a conversa, $q \to p$, nem a inversa,
$\lnot p \to \lnot q$, possuem o mesmo valor lógico que $p \to q$ para todos
valores lógicos possíveis de $p$ e $q$. Note que quando $p$ é verdadeiro e $q$ é
falso, a sentença condicional original é falsa, mas a conversa e a inversa são
ambas verdadeiras. Quando duas proposições compostas possuem o mesmo valor
lógico dissemos que elas são \textbf{equivalentes}.

Para ilustrar a utilização de sentenças condicionais temos o exemplo
\ref{exem17}.


\begin{exmp}
\label{exem17}
Quais são, a contrapositiva, a conversa e a inversa da sentença condicional
\begin{center}\emph{``A equipa da casa vence sempre que chove''}\end{center}
\end{exmp}

\begin{description}
	\item[Solução] Como ``$q$ sempre que $p$'' é uma das formas de expressar uma
	sentença condicional $p \to q$, a sentença original pode ser reescrita como:
	
	\begin{center}\emph{``Se estiver a chover, então a equipa da casa
	vence.''}\end{center}
	
	Consequentemente, a contrapositiva desta sentença condicional é:
	
	\begin{center}\emph{``Se a equipa da casa não vence, então não está a
	chover''.}\end{center}
	
	A conversa é:
	
	\begin{center}\emph{``Se a equipa da casa vence, então está a
	chover''}\end{center}
	
	A inversa é:
	
	\begin{center}\emph{``Se não está a chover, então a equipa da casa não
	vence''.}\end{center}
\end{description}

\begin{defn}
\label{def16}
(Bicondicional) Sejam as proposições $p$ e $q$. A \emph{sentença bicondicional}
$p \leftrightarrow q$ é a proposição ``$p$ se e somente se $q$.'' A sentença
bicondicional $p \leftrightarrow q$ é verdadeira quando $p$ e $q$ possuem os
mesmos valores lógicos, e é falsa no caso contrario. As sentenças
bicondicionais são também chamadas de dupla-implicações.
\end{defn}

A tablela de verdade para $p \leftrightarrow q$ é apresentada na tabela
\ref{tabela:17}. 

\begin{table}[H]
\centering
\begin{tabular}{|c|c|c|}%
\toprule
\textbf{$p$} & \textbf{$q$} & \textbf{$p \leftrightarrow q$}\\ 
\midrule
V &	V & V\\
V &	F & F\\
F &	V & F\\
F &	F & V\\
\bottomrule%
\end{tabular}%
\caption{A tabela de verdade da sentença bicondicional $p \leftrightarrow q$.}
\label{tabela:16}
\end{table}

Note que a expressão $p \leftrightarrow q$ é verdadeira quando ambas sentenças
condicionais $p \to q$ e $q \to p$ são verdadeiras e falsas no caso contrário.
Esta é a razao da utilização das palavras ``se e somente se'' para expressar
esta conexão lógica e também da utilização da combinação dos símbolos $\rightarrow$ e $\leftarrow$. Existem algumas formas comuns de
expressar $p \leftrightarrow q$.
\begin{itemize}
  \item \emph{``$p$ é necessário e suficiente para $q$''}
  \item \emph{``Se $p$ então $q$, e vice-versa''}
  \item \emph{``$p$ sse $q$''.}
\end{itemize}

Note que $p \leftrightarrow q$ possui o mesmo valor lógico que $(p \to q) \land
(q \to p)$.

\label{exem18}
\begin{exmp}
Seja $p$ a sentença \emph{``Podes viajar''} e $q$ a sentença \emph{``Compras o
bilhete''}.
Então $p \leftrightarrow q$ é a expressão \emph{``Podes viajar se e somente se
compras o bilhete''}. Esta sentença é verdadeira se $p$ e $q$ são ambos
verdadeiros ou ambos falsos, isto é, se compras um bilhete e podes viajar ou se
não compras o bilhete e não podes viajar. É falsa quando $p$ e $q$ possuem valores lógicos
opostos.
\end{exmp}

\subsection*{\underline{Tabelas de Verdade de Proposições Compostas}}

Até agora estudamos 4 conectores lógicos importantes - conjunções, disjunções,
sentenças condicionais, e sentenças bicondicionais - e também as negações.
Podemos utilizar estes conectores para construír proposições compostas complexas
que envolvem um elevado número de variáveis proposicionais. Podemos utilizar
tabelas de verdade para determinar os valores lógicos destas proposições
compostas, como o exemplo abaixo ilustra. Utilizamos uma coluna separada para
encontrar o valor lógico de cada expressão composta que ocorre na proposição
composta. Os valores lógicos da proposição composta para cada combinação de
valores lógicos das variáveis proposicioanis nesta expressão é encontrado na
última coluna da tabela.

\label{exem19}
\begin{exmp}
Construa a tabela de verdade para a proposição composta \begin{center}$(p \lor
\lnot q) \to (p \land q).$\end{center}
\end{exmp}

\begin{description}
\item[Solução] Como esta tabela involve duas variáveis proposicionais $p$ e $q$,
teremos quatro linhas nesta tabela de verdade, uma para cada par de valores
lógicos VV, VF, FV, FF. As primeiras duas colunas são utilizadas para para os
valores lógicos de $p$ e $q$, respectivamente. Na terceira coluna encontramos o
valor lógico $\lnot q$, necessários para encontrar o valor lógico de $p \lor
\lnot q$, encontrados na quarta coluna. A quinta coluna fornece os valores
lógicos de $p \land q$. Finalmente, o valor lógico de $(p \lor \lnot q) \to (p
\land q)$ é encontrado na última coluna. A tabela de verdade resultante é:\\

\begin{table}[H]
	\centering
	\begin{tabular}{|c|c|c|c|c|c|}%
	\toprule
	\textbf{$p$} & \textbf{$q$}	& \textbf{$\lnot q$} & \textbf{$p \lor \lnot q$} &
	\textbf{$p \land q$} & \textbf{$(p \lor \lnot q) \to (p \land q)$}\\
	\midrule
	V &	V & F & V & V & V\\
	V &	F & V & V & F & F\\
	F &	V & F & F & F & V\\
	F &	F & V & V & F & F\\
	\bottomrule%
	\end{tabular}%
	\caption{Tabela de verdade $(p \lor q) \to
(p \land q)$.}
	\label{tabela:17}
\end{table}
\end{description}

%precedencIAS
\subsection*{\underline{Precedência dos operadores lógicos}}

Podemos construír proposições compostas utilizando o operador de negação e os
operadores lógicos definidos até agora. No geral, utilizaremos os parêntesis
para especificar a ordem na qual os operadores lógicos numa proposição composta
serão aplicados. Por exemplo, $(p \lor q) \land (\lnot r)$ é a conjunção de $p
\lor q$ e $\lnot r$. No entanto, para reduzir o número de parêntesis,
especificamos que o operador de negação é aplicado antes dos outros operadores
lógicos. Isto significa que $\lnot p \land q$ é a conjunção de $\lnot p$ e $q$,
nomeadamente, $(\lnot p) \land \lnot q$, e não a negação da conjunção de $p$ e
$q$, nomeadamente, $\lnot (p \land q)$.

Outra regra geral das precedências é que o operador de conjunção possui
precedência sobre o operador de disjunção, tal que $p \land q \lor r$ significa
$(p \land q) \lor r$ em vez de $p \land (q \lor r)$. Como esta regra é difícil
de lembrar, iremos utilizar parêntesis para que a ordem dos operadores
de disjunção e conjunção seja clara.

Finalmente, é uma regra aceite que os operadores de condicional e
bicondicional, $\to$ e $\leftrightarrow$, possuem baixa precedência com relação
os operadores de conjunção e disjunção. Consequentemente, $p \lor q \to r$ é o
mesmo que $(p \lor q) \to r$. Iremos utilizar parentêsis quando a ordem do
operador condicional e bicondicional estiver em questão, embora o operador
condicional tenha maior prioridade que o operador bicondicional. A tabela
\ref{tabela:18} apresenta os níveis de precedência dos operadores lógicos,
$\lnot, \land, \lor$ e $\leftrightarrow$.

	
\begin{table}[H]
	\centering
	\begin{tabular}{|c|c|c|}%
	\toprule
	\textbf{Operador} & \textbf{Precedência}\\
	\midrule
	$\lnot$ & $1$ \\
	$\land$ & $2$ \\
	$\lor$ & $3$ \\
	$\to$ & $4$ \\
	$\leftrightarrow$ & $5$ \\
	\bottomrule%
	\end{tabular}%
	\caption{Precedência dos operadores lógicos.}
	\label{tabela:18}
\end{table}


\section{Equivalências proposicionais}
\label{sec:eqproposicioanis}

Um passo muito importante utilizado num argumento matemático é a substituição de
uma sentença por outra com o mesmo valor lógico. Por causa disso, métodos que
produzem proposições com o mesmo valor lógico que uma dada proposição composta
são extensivamente utilizados na construção de argumentos matemáticos. Note que
iremos utilizar o termo ``proposição composta'' para nos referir à uma expressão
formada a partir de variáveis proposicionais utilizando operadores lógicos, como
$p \land q$.

Começamos por classificar as proposições compostas de acordo aos seus valores
lógicos possíveis.

\label{def17}
\begin{defn} (Tautologia, Contradição e Contingência) Uma proposição composta
que é sempre verdadeira, não importando valor lógicos das variáveis
proposicionais que a compoêm, é chamada de \emph{tautologia}. Uma proposição
composta que é sempre falsa é chamada de \emph{contradição}. A proposição
composta que não é nem uma tautologia nem uma contradição é chamada de
\emph{contingência}.
\end{defn}

Tautologias e contradições são geralmente importantes no raciocínio matemático.
O exemplo \ref{exem120} ilustra estes dois tipos de proposições compostas.


\begin{exmp}
\label{exem120}
Podemos construir exemplos de tautologias e contradições utilizando apenas uma
variável proposicional. Considere as tabelas de verdade de $p \lor \lnot p$ e
$p \land \lnot p$, apresentadas na tabela \ref{tabela:19}. Como $p \lor \lnot p$ é
sempre verdadeiro, é uma tautologia. Como $p \land \lnot p$ é sempre falso, é uma contradição.
\end{exmp}

\begin{table}[H]
	\centering
	\begin{tabular}{|c|c|c|c|}%
	\toprule
	\textbf{$p$} & \textbf{$\lnot p$} & \textbf{$p \lor \lnot p$} & \textbf{$p
	\land \lnot p$}\\
	\midrule
	V & F & V & F\\
	F & V & V & F\\
	\bottomrule%
	\end{tabular}%
	\caption{Exemplo de uma Tautologia e de uma Contradição.}
	\label{tabela:19}
\end{table}

\subsection*{\underline{Equivalências Lógicas}}

Proposições compostas que possuem o mesmo valor lógico em todos os casos
possíves são chamadas de \textbf{lógicamente equivalentes}. Também podemos
definir este conceito da seguinte forma.

\label{def18}
\begin{defn}
As proposições compostas $p$ e $q$ são chamadas de \emph{lógicamente
equivalentes} se $p \leftrightarrow q$ é uma tautologia. A notação $p \equiv q$
denota que $p$ e $q$ são lógicamente equivalentes.
\end{defn}

\begin{description}
\item[Nota:] O símbolo $\equiv$ não é um conector lógico, e $p \equiv q$ não é
uma proposição composta, mas sim a sentença que diz que $p \leftrightarrow q$ é
uma tautologia. O símbolo $\iff$ é algumas vezes uilizado em detrimento de
$\equiv$ para denotar equivalências lógicas.
\end{description}

Uma forma de determinar se duas proposições compostas são equivalentes é uma
utilizar uma tabela de verdade. Em particular, as proposições compostas $p$ e
$q$ são equivalentes se e sómente se os valores lógicos nas suas colunas são
iguais. O exemplo \ref{exem121} ilustra este método por estabelecer uma
equivalência lógica extremamente importante e útil, nomeadamente, de $\lnot
(p\lor q)$ com $\lnot p \land \lnot q$. Esta equivalência lógica é uma das
\textbf{leis de De Morgan}, apresentadas na tabela \ref{tabela:121}, em homenagem
ao matemático Augustus De Morgan, do século 19.


\begin{exmp}
\label{exem121}
Mostre que $\lnot (p \land q)$ e $\lnot p \land \lnot q$ são lógicamente
equivalentes.
\end{exmp}

\begin{description}
\item[Solução] A tabela de verdade para estas proposições compostas é
apresentada na tabela \ref{tabela:120}. Como os valores lógicos das proposições
compostas $\lnot (p \land q)$ e $\lnot p \land \lnot q$ são iguais para todas as
combinações possíveis de $p$ e $q$, concluímos que $\lnot (p and q)
\leftrightarrow \lnot p \land \lnot q$ é uma tautologia e que estas proposições
compostas são lógicamente equivalentes.
\end{description}

\begin{table}[H]
	\centering
	\begin{tabular}{|c|c|c|c|c|c|c|}%
	\toprule
	\textbf{$p$} & \textbf{$q$} & \textbf{$p \lor q$} & \textbf{$\lnot (p
	\lor q)$} & \textbf{$\lnot p$}	& \textbf{$\lnot q$}	&	\textbf{$\lnot p
	\land \lnot q$}\\
	\midrule
	V & V & V & F & F & F & F\\
	V & F & V & F & F & V & F\\
	F & V & V & F & V & F & F\\
	F & F & F & V & V & V & V\\
	\bottomrule%
	\end{tabular}%
	\caption{Tabela de verdade de $\lnot (p \land q)$ e $\lnot p \land \lnot q$.}
	\label{tabela:120}
\end{table}

\begin{table}[H]
	\centering
	\begin{tabular}{|c|}%
	\toprule
	$\lnot (p \land q) \equiv \lnot p \lor \lnot q$\\
	$\lnot (p \lor q) \equiv \lnot p \land \lnot q$\\
	\bottomrule%
	\end{tabular}%
	\caption{Leis de De Morgan.}
	\label{tabela:121}
\end{table}



\begin{exmp}
\label{exem122}
Mostre que $p \to q$ e $\lnot p \lor q$ são lógicamente equivalentes.
\end{exmp}

\begin{description}
\item[Solução] Construímos a tabela de verdade para estas proposições
compostas na tabela \ref{tabela:122}. Como os valores lógicos de $\lnot p \lor
q$ e $p \to q$ são iguais, elas são lógicamente equivalentes.
\end{description}

\begin{table}[H]
	\centering
	\begin{tabular}{|c|c|c|c|c|}%
	\toprule
	\textbf{$p$} & \textbf{$q$} & \textbf{$\lnot p$} & \textbf{$\lnot p
	\lor q$} & \textbf{$p \to q$}\\
	\midrule
	V & V & F & V & V\\
	V & F & F & F & F\\
	F & V & V & V & V\\
	F & F & V & V & V\\
	\bottomrule%
	\end{tabular}%
	\caption{Tabelas de verdade de $\lnot p \lor q$ e $p \to q$.}
	\label{tabela:122}
\end{table}

\label{exem123}
\begin{exmp}
Mostre que $p \lor (q \land r)$ e $(p \lor q) \land (p \lor r)$ são lógicamente
equivalentes. Esta é a \emph{lei distributiva} da disjunção sobre a conjunção.
\end{exmp}

\begin{description}
\item[Solução:] Construímos a tabela de verdade destas proposições compostas na
tabela \ref{tabela:123}. Como os valores lógicos de $p \lor (q \land r)$ e $(p
\lor q) \land (p \lor r)$ iguais, estas duas proposições compostas são
lógicamente equivalentes.
\end{description}

\begin{table}[H]
	\centering
	\begin{tabular}{|c|c|c|c|c|c|c|c|}%
	\toprule
	\textbf{$p$} & \textbf{$q$} & \textbf{$r$} & \textbf{$q \land r$} &
	\textbf{$p \lor (q \land r)$} & \textbf{$p \lor q$} & \textbf{$p \lor r$} &
	\textbf{$(p \lor q) \land (p \lor r)$}\\
	\midrule
	V & V & V & V & V & V & V & V\\
	V & V & F & F & V & V & V & V\\
	V & F & V & F & V & V & V & V\\
	V & F & F & F & V & V & V & V\\
	F & V & V & V & V & V & V & V\\
	F & V & F & F & F & V & F & F\\
	F & F & V & F & F & F & V & F\\
	F & F & F & F & F & F & F & F\\
	\bottomrule%
	\end{tabular}%
	\caption{Tabelas de verdade de $p \lor (q \land r)$ e $(p \lor q) \land (p \lor
	r)$.}
	\label{tabela:123}
\end{table}
%\section{Lógica de predicados}
%\label{sec:logicadepredicados}
A tabela \ref{tabela:124} contém algumas equivalências importantes. Nestas
equivalências, \textbf{V} denota a proposição composta que é sempre verdadeira e
\textbf{F} denota a proposição composta que é sempre falsa.

\begin{table}[H]
	\centering
	\begin{tabular}{|l|l|}%
	\toprule
	\textbf{Equivalência} & \textbf{Nome}\\
	\midrule
	$p \land \textbf{V} \equiv p$ &	Leis da Identidade\\
	$p \lor \textbf{F} \equiv p$ &	\\
	\midrule
	$p \lor \textbf{V} \equiv \textbf{V}$ &	Leis da Dominância\\
	$p \land \textbf{F} \equiv \textbf{F}$ &\\
	\midrule
	$p \lor p \equiv p$ &	Leis da Idempotência\\
	$p \land p \equiv p$ &	\\
	\midrule
	$\lnot (\lnot p) \equiv p$ & Lei da Dupla Negação\\
	\midrule
	$p \lor q \equiv q \lor p$ & Leis Comutativas\\
	$p \land q \equiv q \land p$ &\\
	\midrule
	$(p \lor q) \lor r \equiv p \lor (q \lor r)$ & Leis Associativas\\
	$(p \land q) \land r \equiv p \land (q \land r)$ &\\
	\midrule
	$p \lor (q \land r) \equiv (p \lor q) \land (p \lor r)$ & Leis Distributivas \\
	$p \land (q \lor r) \equiv (p \land q) \lor (p \land r)$ & \\
	\midrule
	$\lnot (p \land q) \equiv \lnot p \lor \lnot q$ & Leis de De Morgan\\
	$\lnot (p \lor q) \equiv \lnot p \land \lnot q$ &\\
	\midrule
	$p \lor (p \land q) \equiv p$ & Leis da Absorção\\
	$p \land (p \lor q) \equiv p$ &\\
	\midrule
	$p \lor \lnot p \equiv \textbf{V}$ & Leis da Negação\\
	$p \land \lnot p \equiv \textbf{F}$ &	\\
	\bottomrule%
	\end{tabular}%
	\caption{Equivalências Lógicas.}
	\label{tabela:124}
\end{table}

Apresentamos também algumas equivalências úteis que são compostas por
proposições compostas que utilizam sentenças condicionais e bicondicionai nas
tabelas \ref{tabela:125} e \ref{tabela:126}, respectivamente. A verificação das
mesmas fica como exercício para o estudante.

\begin{table}[H]
	\centering
	\begin{tabular}{|l|}%
	\toprule
	$p \to q \equiv \lnot p \lor q$\\
	$p \to q \equiv \lnot q \to \lnot p$\\
	$p \lor q \equiv \lnot p \to q$\\
	$p \land q \equiv \lnot (p \to \lnot q)$\\
	$\lnot (p \to q) \equiv p \land \lnot q$\\
	$(p \to q) \land (p \to r) \equiv p \to (q \land r)$\\
	$(p \to r) \land (q \to r) \equiv (p \lor q) \to r$\\
	$(p \to q) \lor (p \to r) \equiv p \to (q \lor r)$\\
	$(p \to r) \lor (q \to r) \equiv (p \land q) \to r$\\
	\bottomrule%
	\end{tabular}%
	\caption{Equivalências Lógicas envolvendo sentenças condicionais.}
	\label{tabela:125}
\end{table}

\begin{table}[H]
	\centering
	\begin{tabular}{|l|}%
	\toprule
	$p \leftrightarrow q \equiv (p \to q) \land (q \to p)$\\
	$p \leftrightarrow q \equiv \lnot p \leftrightarrow \lnot q$\\
	$p \leftrightarrow q \equiv (p \land q) \lor (\lnot p \land \lnot q)$\\
	$\lnot (p \leftrightarrow q) \equiv p \leftrightarrow \lnot q$\\
	\bottomrule%
	\end{tabular}%
	\caption{Equivalências Lógicas envolvendo sentenças bicondicionais.}
	\label{tabela:126}
\end{table}
%Comments can be added to the margins of the document using the \todo{Here's a comment in the margin!} todo command, as shown in the example on the right. You can also add inline comments:

%\todo[inline, color=green!40]{This is an inline comment.}
\subsection*{\underline{Utilizando as leis de De Morgan}}

As duas equivalências lógicas conhecidas como as leis de De Morgan são
particularmente importantes. Elas nos dizem como negar conjunções e disjunções.
Em particular, a equivalência $\lnot (p \lor q) \equiv \lnot p \land \lnot q$
nos diz que a negação de uma disjunção é formada pela conjunção das negações das
proposições que a compoêm. Similarmente, a equivalência $\lnot (p \land q)
\equiv \lnot p \lor \lnot q$ nos diz que a negação da conjunção é formada pela
disjunção das negações das proposições que a compoêm. O exemplo \ref{exem124}
ilustra a utilização das leis de De Morgan.


\begin{exmp}
\label{exem124}
Utilize as leis de De Morgan para expressar as negações de ``O Miguel possui um
telemóvel e um laptop'' e ``A Manuela irá ao concerto ou Sérgio irá ao
concerto''.
\end{exmp}

\begin{description}
\item[Solução: ]Seja $p$ a sentença ``O Miguel possui um telemóvel'' e $q$ a
sentença ``Miguel possui um laptop''. Então a sentença ``O Miguel possui um
telemóvel e um laptop'' pode ser representada por $p \land q$. Pela primeira lei
de De Morgan, $\lnot (p \land q)$ é equivalente a $\lnot p \lor \lnot q$.
Consequentemente, podemos expressar a negação da sentença original como ``o
Miguel não possui um telemóvel ou não possui um laptop''.
\item[]Seja $r$ ``A Manuela irá ao concerto'' e $s$ ``Sérgio irá ao  concerto''.
Então ``A Manuela irá ao concerto ou Sérgio irá ao concerto'' pode ser
representada por $r \lor s$. Pela segunda lei de De Morgan, $\lnot (r \lor s)$ é
equivalente a $\lnot r \land \lnot s$. Consequentemente, podemos expressar a
negação da nossa sentença original como ``A Manuela não irá ao concerto e o
Sérgio não irá ao concerto''.
\end{description}


\subsection*{\underline{Construíndo Novas Equivalências Lógicas}}

As equivalências lógicas apresentadas na tabela \ref{tabela:124} e outras que
foram estabelecidas (como as das tabelas \ref{tabela:125} e \ref{tabela:126}),
podem ser utilizadas para construír novas equivalências lógicas. A justificação
disto é que uma proposição numa proposição composta pode ser substituída por
outra proposição composta que seja lógicamente equivalente sem alterar os
valores lógicos na proposição composta original. Esta técnica é ilustrada nos
exemplos \ref{exem125} à \ref{exem126}, onde também utilizamos o facto de que se
$p$ e $q$ são logicamente equivalentes e $q$ e $r$ são lógicamente equivalentes,
então, $p$ e $r$ são logicamente equivalentes.

\begin{exmp}
\label{exem125}
Mostre que $\lnot (p \to q)$ e $p \land \lnot q$ são lógicamente equivalentes.
\end{exmp}

\begin{description}
\item[Solução:] Poderíamos fácilmente utilizar uma tabela de verdade para provar
que estas duas sentenças são lógicamente equivalentes (tal como fizemos
anteriormente). No entanto, queremos ilustrar como utilizar as identidades
lógicas que conhecemos para estabelecer novas identidades lógicas, que é algo de
importância prática no estabelecimento de equivalências lógicas de proposições
compostas com um número elevado de variáveis. Sendo assim, iremos estabelecer
esta equivalência por desenvolver uma série de equivalências lógicas utilizando
as equivalências da tabela \ref{tabela:124} de cada vez, começando por $\lnot
(p \to q)$ e terminando em $p \land \lnot q$. Temos as seguintes equivalências.

 \begin{table}[H]
	\centering
	\begin{tabular}{rcll}%
	$\lnot (p \to q)$ & $\equiv$ & $\lnot (\lnot p \lor q)$ & \emph{pelo exemplo
	\ref{exem122}}\\
	 & $\equiv$ & $\lnot (\lnot p) \land \lnot q$ & \emph{pela segunda lei
	 de De Morgan}\\	
	& $\equiv$ & $p \land \lnot q$ & \emph{pela lei da dupla negação}\\	
	\end{tabular}%
\end{table}
\end{description}


\begin{exmp}
\label{exem126}
Mostre que $\lnot (p \lor (\lnot p \land q))$ e $\lnot p \land \lnot q$ são
lógicamente equivalentes por desenvolver uma série de equivalências lógicas.
\end{exmp}

\begin{description}
\item[Solução:] Iremos utilizar uma das equivalências na tabela \ref{tabela:124}
uma de cada vez, começando por $\lnot (p \lor (\lnot p \land q))$ e terminando
com $\lnot p \land \lnot q$. Teremos as seguintes equivalências.

\begin{table}[H]
	\centering
	\begin{tabular}{rcll}%
	$\lnot (p \lor (\lnot p \land q))$ & $\equiv$ & $\lnot p \land \lnot (\lnot p
	\land q)$ & \emph{pela segunda lei de De Morgan}\\
	 & $\equiv$ & $\lnot p \land [\lnot(\lnot p) \lor \lnot q]$ & \emph{pela
	 primeira lei de De Morgan}\\	
	& $\equiv$ & $\lnot p \land (p \lor \lnot q)$ & \emph{pela lei da dupla
	negação}\\
	& $\equiv$ & $(\lnot p \land p) \lor (\lnot p \land \lnot q)$ & \emph{pela
	segunda lei distributiva}\\
	& $\equiv$ & $\textbf{F} \lor (\lnot p \land \lnot q)$ & \emph{porque $\lnot
	p \land p \equiv \textbf{F}$}\\
	& $\equiv$ & $(\lnot p \land \lnot q) \lor \textbf{F}$ & \emph{pela lei
	comutativa da disjunção}\\
	& $\equiv$ & $\lnot p \land \lnot q$ & \emph{pela lei da identidade para
	\textbf{F}}
	
	\end{tabular}%
\end{table}
\end{description}

\subsection*{\underline{Satisfabilidade Proposicional}}

Um proposição composta é \textbf{satisfazível} se existe uma atribuição de
valores lógicos às suas variáveis que a torna verdadeira. Quando tais
atribuições não existem. isto é, quando a proposição composta é falsa para todas
as atribuições de valores lógicos às variáveis, a proposição composta é
\textbf{insatisfazível}. Note que uma proposição composta é insatisfazível se e
somente se a sua negação é verdadeira para todas as atribuições de valores
lógicos às suas variáveis, isto é, se e somente se a sua negação é uma
tautologia.

Quando encontramos uma atribuição de valores lógicos que torna a proposição
composta verdadeira, demostramos que ela é satisfazível; essa tal atribuição é
chamada de \textbf{solução} deste particula problema de satisfabilidade. No
entanto, para mostrar que uma proposição composta é insatisfazível, devemos
mostrar que \emph{toda} a atribuição de valores lógicos às suas variáveis a
torna falsa. Embora possamos utilizar sempre uma tabela de verdade para
determinar se uma proposição composta é satisfazíl, geralmente é mais eficiente
não utilizar uma, tal como indica o exemplo \ref{exem28}.

\begin{exmp}
\label{exem28}
Determine se cada uma das proposições compostas $(p \lor \lnot q) \land (q \lor
\lnot r) \land (r \lor \lnot p)$, $(p \lor q \lor r) \land (\lnot p \lor \lnot q
\lor \lnot r)$ e $(p \lor \lnot q) \land (q \lor \lnot r) \land (r \lor \lnot p)
\land (p \lor q \lor r) \land (\lnot p \lor \lnot q \lor \lnot r)$ é
satisfazível.
\end{exmp}

\begin{description}
\item[Solução: ]Ao invés de utilizar uma tabela de verdade para resolver este
problema, iremos raciocinar sobre os valores lógicos. Note que $(p \lor \lnot q) \land (q \lor
\lnot r) \land (r \lor \lnot p)$ é verdadeiro quando as três variáveis $p, q$ e
$r$ possuem os mesmos valores lógicos. Assim, a proposição composta é
satisfazível pois existe pelo menos uma atribuição de valores lógicos para $p,
q$ e $r$ que a torna verdadeira. Similarmente, note que $(p \lor q \lor r)
\land (\lnot p \lor \lnot q \lor \lnot r)$ é verdadeira quando pelo menos um
entre $p, q$ e $r$ é verdadeiro e pelo menos um é falso. Assim, $(p \lor q \lor
r) \land (\lnot p \lor \lnot q \lor \lnot r)$ é satisfazível pois existe pelo
menos uma atribuição de valores lógicos para $p, q$ e $r$ que a torna
verdadeira.

Finalmente note que para $(p \lor \lnot q) \land (q \lor \lnot r) \land (r \lor \lnot p)
\land (p \lor q \lor r) \land (\lnot p \lor \lnot q \lor \lnot r)$ ser
verdadeira, $(p \lor \lnot q) \land (q \lor \lnot r) \land (r \lor \lnot p)$ e
$(p \lor q \lor r) \land (\lnot p \lor \lnot q \lor \lnot r)$ devem ambas ser
verdadeiras. Para a primeira ser verdadeira, as três variáveis devem possuir o
mesmo valor lógico, e para a segunda ser verdadeira, pelo menos uma entre três
varáveis deve ser verdadeira e pelo menos uma deve ser falsa. No entanto, estas
condições são contraditórias. Destas observações podemos concluír que nenhuma
atribuição de valores lógicos a $p, q$ e $r$ torna $(p \lor \lnot q) \land (q
\lor \lnot r) \land (r \lor \lnot p) \land (p \lor q \lor r) \land (\lnot p
\lor \lnot q \lor \lnot r)$ verdadeiro. Logo é insatisfazível.

\end{description}

\section{Predicados e Quantificadores}

\subsection*{\underline{Introdução}}
\subsection*{\underline{Predicados}}
\subsection*{\underline{Quantificadores}}
\subsection*{\underline{Quantificadores com Domínios Restrictos}}
\subsection*{\underline{Precedência de Quantificadores}}
\subsection*{\underline{Variáveis Ligadas}}
\subsection*{\underline{Equivalência Lógicas Involvendo Quantificadores}}
\subsection*{\underline{Negação de Expressões Quantificadas}}
\subsection*{\underline{Tradução de Expressões Lógicas}}

\section{Quantificadores Aninhados}

\subsection*{\underline{Sentenças Involvendo Quantificadores Aninhados}}
\subsection*{\underline{A Ordem dos Quantificadores}}
\subsection*{\underline{Sentenças Matemáticas e Sentenças Involvendo Quantificadores
Aninhados}}
\subsection*{\underline{Tradução de Quantificadores Aninhados para Português}}
\subsection*{\underline{Negação de Quantificadores Aninhados}}

\section{Regras de Inferência}
\section{Demonstrações}
\section{Métodos e Estratégias de Demonstração}