\chapter*{Exercícios do Capítulo \ref{cap:logicaformal}}
%%1

\begin{enumerate}
  	\item Quais das seguintes frases são proposições?
  	\begin{enumerate}
	  	\item A lua é feita de queijo verde.
	  	\item Ele é certamente, um homem alto.
	  	\item Dois é um número primo.
	  	\item O jovo vai acabar logo?
	  	\item Os juros vão subir ano que vem.
	  	\item Os juro vão descer ano que vem.
	  	\item $x^2 - 4 = 0.$
  	\end{enumerate}
  	\item Qual das seguintes sentenças é uma proposição? Quais são os valores de verdade para as que são proposições?
  	\begin{enumerate}
	  	  \item Lisboa é a capital do Brasil.
	  	  \item $2 + 3 = 5$
	  	  \item Que horas são?
	  	  \item $2^n \geq 100$
	  	  \item A lua é feita de queijo.
    \end{enumerate} 
    \item Qual é a negação de cada uma das seguintes proposições?
    \begin{enumerate}
	     \item A Joana tem um leitor de MP3.
	     \item Não existe poluição em Luanda.
	     \item $2 + 2 = 4$
	     \item O Paulo e o Tomás são amigos.
	     \item A Maria envia mais de 100 SMS por dia.
  	\end{enumerate}
  	\item Suponha que o telemóvel A tenha 256 MB de RAM e 32 GB de ROM e que a sua resolução gráfica seja de 8 MP; o telemóvel
  	B possui 288 MB de RAM, 64 GB de ROM e 4 MP de resolução gráfica; e por fim o telemóvel C possui 128 MB de RAM, 32 GB de ROM
  	e 5 MP de resolução gráfica. Determine o valor de verdade para cada uma das seguintes proposições.
  	\begin{enumerate}
  		 \item O telemóvel B possui a maior capacidade de RAM.
  		 \item O telemóvel C possui a maior capacidade de ROM ou maior resolução gráfica do que o telemóvel B.
  		 \item O telemóvel B possui mais RAM, mais ROM e mais MP do que o telemóvel A.
  		 \item Se o telemóvel B possui mais RAM e mais ROM que o telemóvel C, então também possui a maior resolução gráfica.
  		 \item O telemóvel A possui mais RAM do que o telemóvel B se e somente se o telemóvel B possui mais RAM do que o telemóvel A.
	\end{enumerate}
	\item Sejam $p$ e $q$ as proposições\\
	\hspace*{1em}$p$: Eu comprei um bilhete para o teatro esta semana\\
	\hspace*{1em}$q$: Eu ganhei um milhão de kwanzas na loteria\\
	Expresse cada uma das seguintes proposições como sentenças em Português.
	\begin{enumerate}
		  \item $\lnot p$
		  \item $p \lor q$
		  \item $p \to q$
		  \item $p \land q$
		  \item $p \leftrightarrow q$
		  \item $\lnot p \to \lnot q$
		  \item $\lnot p \land \lnot q$
		  \item $\lnot p \lor (p \lor q)$
  	\end{enumerate}
  	\item Determine se os seguintes bicondicionais são verdadeiros ou falsos.
  	\begin{enumerate}
  		\item $2+2=4$ se e somente se $1+1=2$.
  		\item $1+1=2$ se e somente se $2+3=4$.
  		\item $1+1=3$ se e somente se macacos conseguem voar.
  		\item $0 > 1$ se e somente se $2>1$.
	\end{enumerate}
  	\item Escreva cada uma das sentenças abaixo na forma ``Se $p$ então $q$''.
  	\begin{enumerate}
  	  \item É necessário lavar o carro do chefe para ser promovido.
  	  \item Quando o vento vem do sul significa que a primaveira aproxima-se.
  	  \item Uma condição suficiente para que a garantia seja válida é a de que o computador foi comprado a menos de um ano.
  	  \item O Pedro é apanhado toda vez que cabula.
  	  \item Obterás acesso ao \emph{website} se pagares a taxa de subscrição.
  	  \item Para ser eleito deves conhecer as pessoas certas.
  	\end{enumerate}
  	\item Construa a tabela de verdade para as seguintes proposições
  	\begin{enumerate}
  	  \item $p \oplus p$ \item $p \oplus \lnot p$ \item $p \oplus \lnot q$ \item $\lnot p \oplus \lnot q$
  	  \item $(p \oplus q) \lor (p \oplus \lnot q)$ \item $(p \oplus q) \land (p \oplus \lnot q)$
  	  \item $((p \to q) \to r) \to s$.
	\end{enumerate}
	\item Explique, sem utilizar uma tabela de verdade, porquê $(p \lor \lnot q) \land (q \lor \lnot r) \land (r \lor \lnot p)$ é 
	verdade quando $p, q$ e $r$ possuem os mesmos valores de verdade e falso em caso contrário.
	\item O soba de uma vila diz que existe um barbeiro numa outra vila muito distante, que apenas faz a barba à pessoas
	e somente à pessoas que não fazem a barba por sí próprias. Será que este barbeiro existe?

	 \item Utiliza a lei de DeMorgan para encontrar a negação para cada uma das seguintes sentenças.
	 \begin{enumerate}
	   	\item O Ndongala vai procurar um emprego ou terminar a licenciatura.
	   	\item A Luísa percebe Java e Estrutura Discreta.
	   	\item O Nadilson é jovem e forte.
	   	\item A Rebecca vai viajar para o Brasil ou para a Espanha.
	 \end{enumerate}
	 \item Utilize tabelas de verdade para verificar as seguintes leis da absorção
	 \begin{enumerate}
	 	\item  $p \lor (p \land q) \equiv p$ \item $p \land (p \lor q) \equiv p$
	\end{enumerate}
	\item Mostre que $(p \to q) \land (q \to r) \to (p \to r)$ é uma tautologia
	\item Encontre uma proposição composta involvendo as variáveis $p, q$ e $r$ que seja verdadeira quando $p$ e $q$ são 
	verdadeiros e $r$ é falso, mas que seja falso em caso contrário.
	\item Encontre uma proposição composta lógicamente equivalente a $p \to q$
	utilizando apenas o operador $\downarrow$.
	\begin{description}
	\item[Nota:] O símbolo $\downarrow$ representa o operador $NOR$ (ou negação do
	operador $\lor$).
	A proposição $p$ $NOR$ $q$ é verdadeira quando ambos $p$ e $q$ são falsos, e
	falsa em caso contrário.
	\end{description}
	\item Demonstre que as seguintes proposições são logicamente equivalentes
	\begin{enumerate}
		\item $p \leftrightarrow q$ e $(p \land q) \lor (\lnot p \land \lnot q)$
		\item $\lnot (p \leftrightarrow q)$ e $p \leftrightarrow \lnot q$
		\item $p \to q) \land (p \to r)$ e $p \to (q \land r)$
	\end{enumerate}

  	\item Cada habitante de uma vila remota sempre diz a verdade ou mente. Um
  habitante desta vila irá apenas responder com ``Sim'' ou ``Não'' a uma
  pergunta feita por um turista. Imagine que és um turista em visita a esta
  cidade e encontras um ponto onde a estrada divide-se em dois caminhos: um que
  leva para o teu destino e outra que leva para uma floresta perigosa. Um
  habitante da vila está mesmo ao lado do ponto da estrada em que te encontras.
  Qual é pergunta que deves fazer a este habitante para obteres a resposta que
  desejas sobre o caminho a tomar para chegar ao teu destino?
  	\item Ao planear uma festa pretendes identificar a quem convidar. No grupo de
  amigos que queres convidar, existem três amigos esquisitos. Sabes que se a
  Joana for, ela não vai ficar contente se o Samuel for. Samuel irá a festa
  apenas se a Katia também for e a Kátia não irá a festa ao menos que Joana
  também não vá. Entre estes três amigos quais deves convidar para garantir que
  ninguém fique infeliz.

  	\item Seja $P(x)$ a expressão ``a palavra $x$ contém a letra $a$''. Quais são os valores de verdade para?
  	\begin{enumerate}
    	\item $P(laranja)$ \item $P(limão)$ \item $P(verdade)$ \item $P(dificil)$ \item $P(falso)$
    \end{enumerate}
    \item Seja $P(x)$ a expressão ``$x$ perde mais do que cinco horas por dia no Facebook'', onde o domínio para $x$ consiste
    em todos os estudantes. Descreve cada uma das expressões a seguir em Português.
    \begin{enumerate}
    	\item $\exists xP(x)$ \item $\forall xP(x)$ \item $\exists x \lnot P(x)$ \item $\forall x \lnot P(x)$ 
	\end{enumerate}
	\item Traduza as seguintes expressões para Português, onde $C(x)$ é ``$x$ é um comediante'' e $E(x)$ é ``$x$ é muito 
	engraçado'' e o domínio que consiste em todas as pessoas.
	\begin{enumerate}
		\item $\forall x(C(x) \to E(x))$ \item $\forall x(C(x) \land E(x))$ 
		\item $\exists x(C(x) \to E(x))$ \item $\exists x(C(x) \land E(x))$
	\end{enumerate}
	\item Seja $Q(x)$ a expressão ``$x+1 > 2$''. Se o domínio consiste em todos os números inteiros, quais são os valores
	de verdade para
	\begin{enumerate}
		\item $Q(0)$ \item $Q(-1)$ \item $Q(1)$ 
		\item $\exists xQ(x)$ \item $\forall xQ(x)$ 
		\item $\exists x \lnot Q(x)$ \item $\forall x \lnot Q(x)$
	\end{enumerate}
	
	\item Determine os valores de verdade para cada uma das seguintes afirmações se o domínio de todas as variáveis consiste
	no conjunto dos números inteiros.
	\begin{enumerate}
		\item $\forall n(n^2 \geq 0)$ \item $\exists n (n^2 = n)$ \item $\forall n(n^2 \geq n)$ \item $\exists n(n^2 < 0)$
	\end{enumerate}
	
	\item Suponha que o domínio das funções proposicionais $P(x)$ consiste nos inteiros $0, 1, 2, 3$ e $4$. Reescreva cada uma
	das proposições utilizando disjunções, conjunnções e negações.
	\begin{enumerate}
		\item $\exists xP(x)$ \item $\forall xP(x)$ \item $\exists x \lnot P(x)$ \item $\forall x \lnot P(x)$
		\item $\lnot \exists xP(x)$ \item $\lnot \forall x P(x)$
	\end{enumerate}
	
	\item Para as seguintes afirmações, encontre um domínio de tal forma que a afirmação seja verdadeira e um domínio de tal forma
	que a afirmação seja falsa.
	\begin{enumerate}
		\item Todo o mundo está a estudar Estruturas Discretas.
		\item Todos têm mais de 21 anos.
		\item Duas pessoas diferentes não possuem a mesma avó.
		\item Todo mundo fala Japonês.
		\item Alguém conhece mais do que duas pessoas.
	\end{enumerate}
	
	\item Traduza as seguintes afirmações em expressões lógicas utilizando predicados, quantificadores e conectores lógicos.
	\begin{enumerate}
	  \item Ninguém é perfeito.
	  \item Nem todo o mundo é perfeito.
	  \item Todos os teus amigos são perfeitos.
	  \item Pelo menos, um dos teus amigos é perfeito.
	  \item Alguém na tua sala foi nascido no século 21.
	  \item Tudo está no sítio certo e em perfeitas condições.
	\end{enumerate}
	
	\item Determine se $\forall x(P(x) \to Q(x))$ e $\forall xP(x) \to \forall
	xQ(x)$ são lógicamente equivalentes. Justifique a sua resposta.
	\item Seja $G(x)$ a expressão ``$x$ possui um gato'', seja $C(x)$ a expressão
	``$x$ possui um cão'' e seja $F(x)$ a expressão ``$x$ possui um furão''.
	Expresse cada uma das seguintes sentenças em termos de $G(x), C(x), F(x)$,
	quantificadores e conectores lógicos. O domínio é o conjuntos dos estudantes da
	tua sala.
	\begin{enumerate}
		\item Um estudante na tua sala possui um gato, um cão e um furão.
		\item Todos os estudantes na tua sala possuem um gato, um cão ou um furão.
		\item Alguns estudantes na tua sala possuem um gato e um furão, mas não
		possuem um cão.
		\item Nenhum estudante na tua sala possui um gato, um cão e um furão.
	\end{enumerate}
	\item Seja $P(x)$ a sentença ``$x = x^2$.'' Se o domínio consiste dos números
	inteiros, qual é o valor lógico das seguintes expressões:
	\begin{enumerate}
		\item $P(0)$\item $P(1)$\item $P(2)$\item $P(-1)$\item$\exists
		xP(x)$\item$\forall P(x)$
	\end{enumerate}
	\item Traduza cada uma das seguintes sentenças em expressões lógicas utilizando
	predicados, quantificadores e conectores lógicos.
	\begin{enumerate}
		\item Ninguém é perfeito.
		\item Todos os teus amigos são perfeitos.
		\item Pelo menos um dos teus amigos é perfeito.
		\item Todos são teus amigos e são perfeitos.
	\end{enumerate}
	\item Seja $Q(x,y)$ a sentença ``$x$ enviou um email a $y$'', onde o domínio de
	$x$ e $y$ consiste em todos os estudantes da tua sala. Expresse cada uma das
	seguintes quantificações em Português.
	\begin{enumerate}
		\item $\exists x\exists yQ(x,y)$
		\item $\exists x\forall yQ(x,y)$
		\item $\forall x\exists yQ(x,y)$
		\item $\exists y\forall xQ(x,y)$
	\end{enumerate}
	\item Encontre um contra-exemplo, quando possível, para estas quantificações
	universais, onde o domínio das variáveis consiste no conjunto dos números
	inteiros.
	\begin{enumerate}
	  \item $\forall x\exists y(x = 1/y)$
	  \item $\forall x\exists y(y^2 - x < 100)$
	  \item $\forall x\forall y(x^2 \centernot{=} y^3)$
  	\end{enumerate}
\end{enumerate}

\newpage
\section*{Exercícios com Programação}

Escreva programas de computador com as entradas e saídas especificadas.
\begin{enumerate}
  \item Dado os valores lógicos das proposições $p$ e $q$, encontre os valores
  lógicos da conjunção, disjunção inclusiva, disjunção exclusiva, condicional e
  a bicondicional destas proposições.
  \item Dada uma proposição composta, determinar se é satisfazível por verificar
  os seus valores lógicos para todos as atribuições positivas de valores lógicos
  às suas variáveis proposicionais.
\end{enumerate}