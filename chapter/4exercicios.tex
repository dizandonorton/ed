\chapter*{Exercícios}
%%1

\section*{Relações e suas propriedades}

\begin{enumerate}
  	\item {Liste os pares ordendados na relação $R$ de $A = \{0,1,2,3,4\}$ para $B = \{0,1,2,3\}$ onde $(a,b) \in R$ se
  	e somente se}
  	\begin{enumerate}
  	  	\item $a = b$ \item $a + b = 4$ \item $a > b$ \item $a | b$ \item $mdc(a,b)=1$ \item $mmc(a,b)=2$
  	\end{enumerate}
  	
  	\item{Para cada uma das relações no conjunto $\{1,2,3,4\}$ indiqye se são: reflexivas, simétricas, antissimétricas
  	e transitivas}
  	\begin{enumerate}
  	  	\item $\{(2, 2), (2, 3), (2, 4), (3, 2), (3, 3), (3, 4)\}$
  	  	\item $\{(1, 1), (1, 2), (2, 1), (2, 2), (3, 3), (4, 4)\}$
  	  	\item $\{(2, 4), (4, 2)\}$
  	  	\item $\{(1, 2), (2, 3), (3, 4)\}$
  	  	\item $\{(1, 1), (2, 2), (3, 3), (4, 4)\}$
  	  	\item $\{(1, 3), (1, 4), (2, 3), (2, 4), (3, 1), (3, 4)\}$
  	\end{enumerate}

  	\item {Determine se a relação $R$ conjunto de todas as pessoas é reflexiva, simétrica, antissimétrica, e/ou
  	transitiva, onde $(a,b) \in R$ se e somente se}
  	\begin{enumerate}
  		\item $a$ é mais alto que $b$.
  		\item $a$ e $b$ foram nascidos no mesmo dia.
  		\item $a$ possui o mesmo apelido que $b$.
  		\item $a$ e $b$ possuem o mesmo avó.
  	\end{enumerate}	
\end{enumerate}

%\section*{Representação de relações}

%\section*{Exercícios com programação}

\vspace*{2em}